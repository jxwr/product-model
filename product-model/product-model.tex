\documentclass[a4paper]{article}
\usepackage{xeCJK}
\usepackage[a4paper,top=2cm,bottom=2cm,left=1.5cm,right=1.5cm,marginparwidth=1.75cm]{geometry}

\title{商品模型形式化定义}
\author{jiaoxiang@meituan.com}
\date{\today}
 
\begin{document}
 
\maketitle

\section{定价维度}
上单和下单是由一组维度$D=\{d_0, d_1, d_2,\ldots, d_n\}$决定的\textbf{最小定价单元}的逆过程。
\begin{itemize} 
    \item 供应链根据$D$定义价格。
    \item 交易根据$D$定位价格。
\end{itemize}
可以认为,订单项一定直接或间接包含了$D$的全维度数据及对应的价格。

\subsection{维度的两部分}
$D$可以分为两部分,一部分定义了产品,剩余部分定义产品的价格,这里将
\begin{itemize}
    \item 定义产品的维度记为$D_{product} \subset D$,其值域为$V_{product}$
    \begin{itemize}
        \item $V_{product} = \{(v_0, v_1, v_2, \ldots, v_n) \mid v_i \in Z_i\}$,这里$Z_i$为$d_i \in D_{product}$的值域
    \end{itemize}
    \item 定义产品价格的维度记为$D_{pricing} = D - D_{product}$,其值域为$V_{pricing}$
\end{itemize}
\begin{equation}
    \{\underbrace{%
        \overbrace{\mathstrut d_0,\ldots,d_k}^{D_{product}},
        \overbrace{\mathstrut d_{k+1},\ldots,d_n}^{D_{pricing}}}
    _{D} \}
\end{equation}
实现时,通常会为$v \in V_{product}$创建一个$id_{product}$,该$id$在进入产品详情页时已经确定。$D_{pricing}$的数据在详情页及填单页由用户选择或填写决定,最终记录在订单项上。

\section{商品结构}
为产品生成一个$id_{product}$比较容易理解,对于$D_{pricing}$要复杂一些。

\subsection{传统零售电商}
实物零售电商定价比较直接,会为每个$v \in V_{pricing}$创建一个$id_{pricing}$
\begin{itemize}
    \item 业界把由$v$确定的实体叫做\textbf{SKU},此时$id_{pricing}=id_{sku}$
    \item $(id_{product}, id_{sku})$确定一个订单项,记为$(id_{product}, id_{sku}) \longrightarrow orderline$
\end{itemize}
\begin{equation}
    \{\underbrace{%
        \overbrace{\mathstrut d_0,\ldots,d_k}^{D_{product}},
        \overbrace{\mathstrut d_{k+1},\ldots,d_n}^{D_{sku}}}
    _{D} \}
\end{equation}
        
\subsection{服务类电商}
对于服务类电商,有两类情况导致不适合为每个$D_{pricing}$创建一个$id$
\begin{enumerate}
    \item 一些维度d的基数较大,为$D_{pricing}$创建$id$会导致$id$爆炸
    \item 价格是由维度根据计算公式动态确定,无法枚举
\end{enumerate}
这种情况,我们可将$D_{pricing}$拆分成部分:$D_{pricingBase}$和$D_{pricingParams}$:
\begin{itemize}
    \item 实现时会为$D_{pricingBase}$生成$id_{pricingBase}$和数据库记录
    \item $D_{pricingParams}$作为该记录内部的价格因素
    \item $id_{pricingBase}$与用户参数$\overline{params}$共同定位价格
\end{itemize}
\begin{equation}
    \{\underbrace{%
        \overbrace{\mathstrut d_0,\ldots,d_k}^{D_{product}},
        \overbrace{\mathstrut d_{k+1},\ldots,d_l}^{D_{pricingBase}},
        \overbrace{\mathstrut d_{l+1},\ldots,d_n}^{D_{pricingParams}}
    }
    _{D} \}
\end{equation}

\subsection{到店业务分析}
\begin{itemize}
    \item \textbf{美团酒店}
        \begin{itemize}
            \item 酒店$goods$没有区分$D_{product}$和$D_{pricing}$
            \item 而是把一部分$D_{pricing}$中的维度放到了$D_{product}$里,其余作为参数(日期,成人儿童数等)
            \item $D$被划分为$D_{goods}$和$D_{params}$两部分,CRS为每个$v \in V_{goods}$创建一个$id_{goods}$
            \item $id_{goods}$和下单参数$\overline{params}$确定一个订单项,记为$(id_{goods}, \overline{params}) \longrightarrow orderline$
        \end{itemize}
        \begin{equation}
            \{\underbrace{%
                \overbrace{\mathstrut d_0,\ldots,d_k}^{D_{goods}},
                \overbrace{\mathstrut d_{k+1},\ldots,d_n}^{D_{params}}
            }
            _{D} \}
        \end{equation}
    \item \textbf{到餐团购}
        \begin{itemize}
            \item 团购$deal$更简单一些,可认为没有用户参数
            \item 团购没有区分$D_{product}$和$D_{pricing}$,$v \in V$对应一个$id_{deal}$
            \item 用$id_{deal}$就可以确定订单项,$id_{deal} \longrightarrow orderline$
        \end{itemize}
        \begin{equation}
            \{\underbrace{%
                \overbrace{\mathstrut d_0,\ldots,d_n}^{D_{deal}}
            }
            _{D} \}
        \end{equation}        
    \item \textbf{度假}
        \begin{itemize}
            \item 度假有两个层级的产品结构:$product+sku$和$product+package+sku$
            \item $product+sku$与传统零售电商类似,只是把日期从$D_{pricing}$中剥离出来
            \begin{enumerate}
                \item 订单项:$(id_{product}, id_{sku}, param_{date}) \longrightarrow orderline$
                \begin{equation}
                    \{\underbrace{%
                        \overbrace{\mathstrut d_0,\ldots,d_k}^{D_{product}},
                        \overbrace{\mathstrut d_{k+1},\ldots,d_{n-2}}^{D_{sku}},
                        \overbrace{\mathstrut d_{n-1},\ldots,d_n}^{D_{date}}
                    }
                    _{D} \}
                \end{equation}                        
            \end{enumerate}
            \item $product+package+sku$除了把日期从$D_{pricing}$中剥离出来,还将省下的维度拆成$D_{package}$和$D_{sku}$
            \begin{enumerate}      
                \item 一共四段:($D_{product}, D_{package}, D_{sku}, D_{date}$)
                \item 订单项:$(id_{product}, id_{package}, id_{sku}, param_{date}) \longrightarrow orderline$ 
                \begin{equation}
                    \{\underbrace{%
                        \overbrace{\mathstrut d_0,\ldots,d_k}^{D_{product}},
                        \overbrace{\mathstrut d_{k+1},\ldots,d_l}^{D_{package}},                        
                        \overbrace{\mathstrut d_{l+1},\ldots,d_m}^{D_{sku}},
                        \overbrace{\mathstrut d_{m+1},\ldots,d_n}^{D_{date}}
                    }
                    _{D} \}
                \end{equation}  
            \end{enumerate}
        \end{itemize}
    \item \textbf{酒店OTA}
        \begin{itemize}
            \item 国际OTA如Expedia和Booking的直连模型是$product+rateplan$形式,这是标准的$D_{product}$和$D_{pricing}$模式
            \item $product+rateplan$形式,应该是最通用的上单模型,不仅仅是针对酒店,因为它已经包含了全部定价信息
        \end{itemize}
        \begin{equation}
            \{\underbrace{%
                \overbrace{\mathstrut d_0,\ldots,d_k}^{D_{product}},
                \overbrace{\mathstrut d_{k+1},\ldots,d_n}^{D_{rateplan}}
            }
            _{D} \}
        \end{equation}   
\end{itemize}

\section{展示}

对于同一个业务,供应链与C端通常会以相同维度对产品做聚合,便于进行统一操作,但不绝对。例如酒店供应链操作按逻辑房型和goods操作,C端展示按物理房型聚合。将两端的聚合维度分别记为$D_{caggr}$和$D_{baggr}$,
\begin{itemize}
    \item 一般来说:$D_{caggr} \subseteq D_{product}$,并且$D_{baggr} \subseteq D_{product}$
    \item 某些品类有标品(SPU)的概念,记为$D_{spu} \subseteq D_{product}$,剩余维度记为$D_{select}=D_{product}-D_{spu}$
\end{itemize}
\begin{equation}
    \{\underbrace{%
        \overbrace{\mathstrut d_0,\ldots,d_k}^{D_{product}},
        \overbrace{\mathstrut d_{k+1},\ldots,d_n}^{D_{rateplan}}
    }
    _{D} \}
\end{equation}
\begin{equation}
    \{\underbrace{%
        \overbrace{\mathstrut d_0,\ldots,d_s}^{D_{spu}},
        \overbrace{\mathstrut d_{s+1},\ldots,d_k,\ldots,d_n}^{D_{select}}
    }
    _{D} \}
\end{equation}

\end{document}