\documentclass[a4paper]{article}
\usepackage{xeCJK}
\usepackage[a4paper,top=2cm,bottom=2cm,left=2cm,right=2cm,marginparwidth=1.75cm]{geometry}

\begin{document}

\section{前提}

\begin{itemize}
    \item AccessCode(简称Code)用于标识供应商最小售卖单元,用$Code$表示所有可能Code的集合。
    \item 调用供应商预定接口时,需要传递Code对应的信息和附加属性。
    \item 最小售卖单元由供应商的一组数据模型维度描述,记这组维度为$Dim=(d_0,d_1,d_2,\dots,d_n)$
    \item 设由${d_i}$表示维度的值域为$D_i$,则$Dim$所表示的维度组合的值域为$V = \{( v_0,v_1,v_2,\dots,v_n) \mid v_i \in D_i, i = 0..n\}$。
    \item $V_c = \{ \overline v \mid code(\overline v) = c, c \in Code, \overline v \in V\}$,这里函数$code$用于生成Code,根据定义可知$V_c \subseteq V$,即供应商有可能不允许维度的任意组合。
\end{itemize}

\section{ 供应商原始模型(各有不同)}
\paragraph{供应商数据由两部分组成:}
\begin{itemize}
    \item \textbf{静态信息},如酒店信息、房型信息、预定义的RatePlan等。
    \item \textbf{动态信息},价格、库存等。记为:$M_{partner}=M_{dynamic}+M_{static}$。\\
    将动态信息的高维数据打平,按最小粒度间夜拆分,$M_{dynamic}$可抽象为:
    \begin{center}
        $M_{dynamic}=\left|\begin{array}{ccc}
            A_0=(a_{0,0},a_{0,1},a_{0,2}, \dots, a_{0,k}) \\\
            A_1=(a_{1,0},a_{1,1},a_{1,2}, \dots, a_{1,k}) \\\
            \dots \\\
            A_{\infty}=(a_{\infty,0},a_{\infty,1},a_{\infty,2},\dots,a_{\infty,k}) 
        \end{array} \right|$,$A_i$表示间夜粒度产品信息
    \end{center}
    \item 由于数据持续更新,所以\textbf{$M_{dynamic}$本质上是一个Stream}。
    \item 记$A_i$类型为$DailyData$,$M_{static}$类型为$StaticData$,Code的类型为$Code$。$A$表示$A_i$组成的集合。
    \item 由\textbf{间夜粒度动态信息}$a\in \{A_0,A_1,...\}$和$M_{static}$,可以得到唯一的Code,形式化可以记为$\exists F,  \forall a \in A, F(a,M_{static}) = c ,  c\in Code$,即针对每个接入方都可以定出一个函数$F::(DailyData, StaticData) \to Code$,根据间夜数据和静态信息找到对应的Code。
    \item 更进一步:$F$并不是必须的,因为\textbf{$F$和前提中函数$code$功能重复},可以增加一个辅助函数$dimExtract$用于从$(a, M_{static})$中抽取Code相关维度数据$\overline v,\overline v \in V_c$。形式化表达为:
    \begin{center}
        $\exists dimExtract,  \forall a \in A, code(dimExtract(a,M_{static})) = c ,  c\in Code$。
    \end{center}
    \item 该供应商数据对应的Code集合为:
    \begin{center}
        $\{ c | c = F(a, M_{static}), a \in M_{dynamic}\}$。
    \end{center}
    注:$M_{dynamic}$无限大,但是Code集合是有限大小。
\end{itemize}

\section{供给存储模型(统一Cache模型,180天)}
\begin{center}
    $M_{cache} = 
    \left|\begin{array}{ccc}
        Cell_{0,code_0},Cell_{0,code_1},Cell_{0,code_2},...,Cell_{0,code_j} \\\
        Cell_{1,code_0},Cell_{1,code_1},Cell_{1,code_2},...,Cell_{1,code_J} \\\
        \dots \\\
        Cell_{179,code_0},Cell_{179,code_1},Cell_{179,code_2},...,Cell_{179,code_j}
    \end{array} \right|
    = Date \times Code$
\end{center}

\begin{itemize}
    \item $Cell_{date,code}$存储$(Date,Code)$粒度各种属性
    \item $M_{dynamic}$到$M_{cache}$的变换可由如下映射表示:
    \begin{center}
        $Cell_{date, F(A_{date},M_{static})} = modelTrans(A_{date},M_{static})$
    \end{center}
\end{itemize}

\section{解释一下}
\begin{enumerate}
    \item 函数$modelTrans$用于将原始数据转换成Cell的数据结构,即Cell存储的数据为$modelTrans(A_{date},M_{static})$
    \item 函数类型:$modelTrans::(DailyData_{date}, StaticData) \to Cell_{(date,code)}$
    \item Code生成:$code=F(A_{date},M_{static})$
    \item 放一起有:$Cell_{date,code} = modelTrans(A_{date},M_{static})$
\end{enumerate}
\subsection{ 供给模型到售卖模型映射(Code=>GoodsId)}

\begin{itemize}
    \item 供给模型的数据维度:
    \begin{center}
        $Dim_C=\{d_0,d_1,d_2,...,d_n\}$
    \end{center}
    \item 售卖模型的数据维度:
    \begin{center}
        $Dim_S=\{s_0,s_1,s_2,...,s_m\}$
    \end{center}
    自销的话售卖模型就是\textbf{产品中心}的模型。
    \item 新供应商可以支持的条件为:
    \begin{center}
        $\exists T, Dim_C  \stackrel{T}{\longrightarrow} Dim_S$,即可以找到映射函数T,将供给维度映射成售卖维度
    \end{center}
    \item 通常供给方维度仅是售卖单元维度的扩充,这时:
        \begin{center}
            {center}$Dim_C=\{d_0,d_1,d_2,...,d_n\}$,$Dim_S=\{s_0,s_1,s_2,...,s_m\}$
        \end{center}
        其中$n \ge m,Dim_S  \subseteq Dim_C$,映射函数$T$的功能就是按维度聚合。
\end{itemize}

\subsection{ 流程 }
\begin{enumerate}
\item $M_{partner}$可以通过函数$F$(间夜粒度数据映射成Code)和$modelTrans$映射成$M_{cache}$:
\begin{center}
    $M_{partner}\stackrel{(F,modelTrans)}{\longrightarrow} M_{cache}$
\end{center}
\item Code的维度集合$Dim_C$可以通过函数$T$映射成C端售卖维度集合$Dim_S$:
\begin{itemize}
\item $Dim_C  \stackrel{T}{\longrightarrow} Dim_S$
\item $Dim_C $与Code对应, $Dim_S$与GoodsIs对应
\item $Code  \stackrel{T_{code}}{\longrightarrow} GoodsId$,$T_{code}$表示Code到GoodsId的映射方法
\end{itemize}
\item 接入工作,需根据实际情况:
\begin{itemize}
\item 确定$Dim_C$,即Code所对应的维度
\item 定义$dimExtract, code, modelTrans, T_{code}, T_{code}^{-1}$这几个方法
\item $Dim_S$目前就是产品中心模型
\end{itemize}
\item 查询时流程是反向的,将GoodsId映射成Code,然后从$M_{cache}$中拿数据:
\begin{itemize}
\item $GoodsId  \stackrel{T_{code}^{-1}}{\longrightarrow} Code$
\item $M_{cache}[Code][DateRange] = 目标Data$
\item 返回给产品中心前,需将$Data$转成接口定义的Model
\end{itemize}
\end{enumerate}

\end{document}
